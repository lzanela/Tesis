% Codificación y idioma
\usepackage[utf8]{inputenc}
\usepackage[T1]{fontenc}
\usepackage[spanish,es-noquoting,es-noshorthands]{babel}
\usepackage{stmaryrd}

% Notación para algoritmos
\usepackage[linesnumbered,ruled,vlined]{algorithm2e}
\renewcommand{\algorithmcfname}{Algoritmo}           % Caption "Algorithm"
\renewcommand{\listalgorithmcfname}{Lista de Algoritmos} % "List of Algorithms"
\SetKwInput{KwReturn}{Retornar}   % instead of "Return"
\SetKwIF{If}{ElseIf}{Else}{Si}{entonces}{Sino si}{Sino}{fin}
\SetKwFor{For}{Para}{hacer}{fin}
\SetKwFor{While}{Mientras}{hacer}{fin}
\SetKwInput{KwData}{Datos}
\SetKwInput{KwResult}{Resultado}
\SetKwInput{KwInput}{Entrada}
\SetKwInput{KwOutput}{Salida}

% Matemática
\usepackage{amsmath, amssymb, amsthm, mathtools}
\usepackage{bbm}
\usepackage{relsize}

% Gráficos, imágenes e ilustraciones
\usepackage{graphicx}
\usepackage{tikz}
\graphicspath{{imagenes/}}
\usetikzlibrary{decorations.pathreplacing,positioning}
\usepackage{placeins}

% Nombres en español
\addto\captionsspanish{\renewcommand{\proofname}{\textbf{Prueba:}}}
\addto\captionsspanish{\renewcommand{\tablename}{Tabla}}

% Estilo de teoremas (numerados por capítulo; cambia [chapter] por [section] si prefieres)
\theoremstyle{plain}
\newtheorem{theorem}{Teorema}[chapter]
\newtheorem{lemma}[theorem]{Lema}
\newtheorem{proposition}[theorem]{Proposición}
\newtheorem{corollary}[theorem]{Corolario}

\theoremstyle{definition}
\newtheorem{definition}[theorem]{Definición}
\newtheorem{example}[theorem]{Ejemplo}

\theoremstyle{remark}
\newtheorem{remark}[theorem]{Observación}

% --- Anexo ---
\renewcommand{\appendixname}{Anexo}

% --- Datos de la portada ---
\newcommand{\TituloTesis}{Una aproximación al problema de \textit{apportionment}: métodos aleatorizados}
\newcommand{\AutorTesis}{Luca Zanela}
\newcommand{\DirectorTesis}{José Correa Haeussler}
\newcommand{\FechaTesis}{Fecha de presentación: 19 de diciembre de 2025}

% Tablas
\usepackage{booktabs}
\usepackage{array}
\usepackage{multirow}

% Hipervínculos
\usepackage{xcolor}     % permite definir colores
\definecolor{azulPetroleo}{RGB}{40,80,120}
\definecolor{verdeGrisaceo}{RGB}{30,100,90}
\definecolor{azulAcero}{RGB}{50,90,150}
\definecolor{rojoOscuro}{RGB}{170,60,40}
\newcommand{\chgpos}[1]{\textcolor{green!40!black}{(#1)}}
\newcommand{\chgneg}[1]{\textcolor{red!70!black}{(#1)}}

\usepackage[
    colorlinks=true,
    linkcolor=azulPetroleo,
    citecolor=verdeGrisaceo,
    urlcolor=azulAcero
]{hyperref}

% Márgenes
\usepackage{geometry}
\geometry{margin=2.5cm}

% Otros paquetes útiles
\usepackage{enumitem}
\usepackage{csquotes} % Recomendado con biblatex

% Bibliografía
\usepackage[
    backend=biber,
    style=alphabetic,
    citestyle=alphabetic,
    sorting=nyt
]{biblatex}
\addbibresource{referencias.bib}

% Nuevo comando para citar autor + número
\newcommand{\cita}[1]{\parencite{#1}}
