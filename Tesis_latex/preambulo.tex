% Codificación y idioma
\usepackage[utf8]{inputenc}
\usepackage[T1]{fontenc}
\usepackage[spanish,es-noquoting,es-noshorthands]{babel}


% Matemática
\usepackage{amsmath, amssymb, amsthm, mathtools}

% Gráficos e imágenes
\usepackage{graphicx}
\graphicspath{{imagenes/}}

% Nombres en español
\addto\captionsspanish{\renewcommand{\proofname}{Prueba}}

% Estilo de teoremas (numerados por capítulo; cambia [chapter] por [section] si prefieres)
\theoremstyle{plain}
\newtheorem{theorem}{Teorema}[chapter]
\newtheorem{lemma}[theorem]{Lema}
\newtheorem{proposition}[theorem]{Proposición}
\newtheorem{corollary}[theorem]{Corolario}

\theoremstyle{definition}
\newtheorem{definition}[theorem]{Definición}
\newtheorem{example}[theorem]{Ejemplo}

\theoremstyle{remark}
\newtheorem{remark}[theorem]{Observación}

% --- Datos de la portada ---
\newcommand{\TituloTesis}{TITULO DE LA TESIS}
\newcommand{\AutorTesis}{Luca Zanela}
\newcommand{\DirectorTesis}{José Correa Haeussler}
\newcommand{\FechaTesis}{Fecha de presentación: 20 de diciembre de 2025}

% Tablas
\usepackage{booktabs}
\usepackage{array}
\usepackage{multirow}

% Hipervínculos
\usepackage[hidelinks]{hyperref}

% Márgenes
\usepackage{geometry}
\geometry{margin=2.5cm}

% Otros paquetes útiles
\usepackage{enumitem}
\usepackage{csquotes} % Recomendado con biblatex

% Bibliografía
\usepackage[
    backend=biber,
    style=alphabetic,
    citestyle=alphabetic,
    sorting=nyt
]{biblatex}
\addbibresource{referencias.bib}

% Nuevo comando para citar autor + número
\newcommand{\cita}[1]{\parencite{#1}}
