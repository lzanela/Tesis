\section{Introducción al problema de apportionment (\textit{prorateo})}

El problema de \textit{apportionment} (o \textit{prorateo} en español, pero es una palabra un poco
fea así que usaremos su versión en inglés) trata sobre cómo repartir de 
forma "justa" las bancas de un congreso o parlamento según la proporción de votos obtenidos
por distintos partidos políticos, o según la proporción de poblaciones de los distintos estados
o provincias. La definición de qué significa que un sistema sea "justo" tendrá que ver con 
qué tan cercano es a la proporcionalidad exacta, y la definiremos con precisión en la próxima sección.

A un nivel un poco más general, el problema consiste en repartir cierta cantidad predeterminada de objetos discretos, indivisibles,
entre ciertos agentes de acuerdo a proporciones de correspondencia dadas: a cada agente $i$ le corresponde una proporción 
$p_i$ de la cantidad de objetos discretos. Esencialmente, el problema se reduce al redondeo de fracciones: cómo redondear de la forma más "justa" posible
un vector de fracciones (las proporciones de bancas correspondientes a los distintos partidos/estados)
a valores enteros, de manera tal que la suma de las fracciones sea igual a la suma de los enteros
redondeados (en ambos casos, esta suma debe ser igual a la cantidad total de bancas a repartir).

A pesar de que el problema aparenta ser de una naturaleza simple, veremos que tiene una complejidad
intrínseca muy grande, que incluso nos llevará a concluir que la existencia de un sistema "perfecto"
es imposible. Para comenzar con un ejemplo quizás clarificador, pensemos en el siguiente escenario de juguete:
2 partidos que reciben exactamente la mitad de los votos cada uno, entre los cuales hay que repartir únicamente una banca.
¿Cómo se asigna esa banca de la forma más justa posible, y qué significa que esa asignación sea "justa"?
Más allá de la definición que se brinde
del término \textit{justicia} - siempre y cuando no haya preferencia particular por ninguno de los partidos- cualquiera de los candidatos
merecería la banca en igual proporción, y por ende cualquiera de las asignaciones sería igual de justa o injusta.

Otro ejemplo, ubicado en las antípodas, sería el caso en el que la cantidad de bancas a repartir es
exactamente igual a la cantidad de votantes: podría ser el caso de un consejo barrial, por ejemplo, en donde cada miembro tiene representatividad 
total, como ejemplifican \cita{balinski1982fair}.

Haremos un recorrido por la historia de este problema, enunciando su definición
formal, los axiomas solicitados para considerar que un sistema es deseable, coherente o "justo", 
los desarrollos teóricos que se fueron produciendo alrededor de esto (principalmente por Balinski, Young,
Pukhelsheim \cita{balinski1980theory}, \cita{balinski1982fair}, \cita{pukkelsheim2017}). Enunciaremos la definición de los llamados "\textit{métodos de divisor}", junto 
con sus propiedades fundamentales, para luego enunciar y demostrar el teorema de imposibilidad de Balinski-Young.
Esto nos servirá como motivación para definir métodos aleatorizados junto con sus respectivos axiomas deseables. 
Atravesaremos algunos ejemplos de métodos aleatorizados,

Si los métodos aleatorizados son factibles de ser usados en escenarios electorales reales es una 
inquietud que queda para el lector.

%Another category of methods may appeal to the gambling man. As one example, 
%construct a roulette wheel divided into fifty slots, one for each state, the 
%size of each slot being exactly proportional to the population of the state. 
%Spin the wheel and drop a small ball onto it: the state at which it comes to 
%rest4 'wins5 5 and is awarded one seat. Do this 435 times consecutively and the 
%house is apportioned. The method is perfectly unbiased: every state is treated 
%fairly; none can complain that the method discriminates against it. 
%An alternative and perhaps preferable scheme is to compare the quotas of the states 
%and first give to each the whole number in its quota; then use a roulette 
%wheel, with each slot proportional to a state's remainder, to distribute the 
%seats left over. Which of all possible methods is fairest! To answer this question, 
%standards or principles must be formulated by which methods can be evaluated. 
%History itself points to such principles. What are the methods that avoid the population 
%paradox, the Alabama paradox, the new states paradox? What are the methods that 
%always stay within the quota? What are the methods that systematically favor neither 
%the large states at the expense of the small nor the small at the expense of the large? 
%Or, for proportional representation systems, which methods discourage the splintering 
%of large parties into smaller ones? Each of these questions captures a principle of apportionment. 
%They have repeatedly been used in history to judge the merits of competing proposals. 
%In a word, they are the guides to those methods that feel right, that are fair, 
%that are indeed proportional. \cita{balinski1982fair} discusses various methods of apportionment in detail.