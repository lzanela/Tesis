\section{Introducción al problema de \textit{apportionment}}

El \textit{apportionment}\footnote{O "prorrateo" en español, pero usaremos su versión en inglés por ser la 
opción más utilizada en la literatura.} 
trata sobre cómo repartir de 
forma "justa" una cantidad finita de objetos indivisibles entre los miembros de un grupo, de acuerdo
a ciertas proporciones. Generalmente, el intento de respetar estas proporciones provoca que las asignaciones
ideales requieran dividir los objetos en partes fraccionarias, lo cual no está permitido por las condiciones
del problema. El conflicto surge a la hora de intentar "redondear" esas partes fraccionarias, de forma tal que al final se asignen cantidades enteras a cada individuo, 
y no sobren ni falten objetos por asignar.

Para ejemplificar este escenario, se puede pensar en el problema de repartir 5 cabras entre 3 hermanos de forma
proporcional de acuerdo a sus edades: si las edades de los hermanos fuesen de 15, 20 y 25 años, la proporción exacta 
correspondiente a cada hermano sería de $\frac{5}{4}$, $\frac{5}{3}$ y $\frac{25}{12}$\footnote{
    Para cada hermano $h$, el total de cabras que le corresponde se calcula como el total de cabras multiplicado por
    la edad de $h$ sobre la suma de las edades. En el caso del hermano menor, resulta $5 \cdot \frac{15}{15+20+25} = \frac{5}{4}$.
} respectivamente. Ninguna de estas 
cantidades es entera, lo cual puede llegar a tentarnos a fraccionar las cabras, violando las restricciones del problema.
La pregunta que surge entonces es: ¿cómo redondeamos estas proporciones para que se asignen las cabras de la forma más justa posible?
La teoría del \textit{apportionment} trata precisamente sobre esta problemática.

Sin embargo, el escenario que más ha alentado y motivado el desarrollo teórico y el estudio de este problema viene del
mundo de la Teoría de Elección Social (\textit{Social Choice}), que se encarga, en términos generales, de estudiar cómo
tomar decisiones de forma colectiva, típicamente 
haciendo una agregación de preferencias individuales en un único perfil de preferencias colectivo y buscando que esto suceda de la forma más justa posible. 
El resultado más famoso de esta disciplina es el Teorema de Imposibilidad de Arrow, que plantea, no obstante, que a la hora de elegir
colectivamente entre ciertas opciones a partir de los perfiles de preferencias de cada individuo (en donde estos perfiles de preferencias son simplemente
ordenamientos de las opciones), no existe forma de agregarlos en un único perfil de preferencias colectivo de forma que se 
satisfaga una serie de desiderata. Posiblemente el lector recuerde una entrevista a Milei en 2023, en la que la entrevistadora le pregunta si cree en la 
democracia, y este contesta "¿Vos conocés el teorema de imposibilidad de Arrow?" \cita{mileiarrow2023}, insinuando que la democracia no es posible.
Si bien mucha gente ha interpretado el resultado de Arrow como una demostración de la imposibilidad de diseñar un sistema verdaderamente democrático, 
el propio Arrow ha dicho que

\say{[las] implicaciones filosóficas y distributivas de la paradoja de la elección social todavía no están claras.
Ciertamente, no hay una salida simple. Espero que otros tomen esta paradoja como un desafío más que como una barrera
desalentadora.} \cita{suzumura1983rational}, sugiriendo que encontrar formas democráticas de decisión colectiva es un 
reto a la imaginación.
Recordaremos esta reflexión al finalizar el capítulo 2.

Con respecto al problema de \textit{apportionment}, la Teoría de Elección Social lo ha tratado dentro del contexto de distribución 
de bancas parlamentarias, bajo la idea de que el congreso debe ser una representación en pequeña escala de la sociedad, y 
guiándose por el lema "una persona, un voto" \cita{balinski1982fair}.
Desde esta perspectiva, se ha intentado entender cómo repartir de forma "justa" las bancas de un congreso o parlamento 
según la proporción de votos obtenidos por distintos partidos políticos, o según la proporción de poblaciones de los distintos estados o provincias. 
La definición de qué significa que un sistema de asignación de bancas sea "justo" tendrá que ver con 
qué tan cercana es la asignación que provee dicho sistema a la proporcionalidad exacta, y la definiremos con precisión en la próxima sección a 
través de una serie de desiderata planteada por \cita{balinski1980theory} y profundizada por \cita{pukelsheim2017}.

Esencialmente, el problema se reduce al redondeo de fracciones: cómo redondear de la forma más "justa" posible
un vector de fracciones (las proporciones de bancas correspondientes a los distintos partidos/estados)
a valores enteros, de manera tal que la suma de las fracciones sea igual a la suma de los enteros
redondeados (en ambos casos, esta suma debe ser igual a la cantidad total de bancas a repartir).

A pesar de que el problema aparenta ser de una naturaleza simple, veremos que tiene una complejidad
intrínseca considerable, que incluso nos llevará a concluir que la existencia de un sistema "perfecto"
es imposible. Para comenzar con un ejemplo quizás clarificador, pensemos en el siguiente escenario de juguete:
2 partidos que reciben exactamente la mitad de los votos cada uno, entre los cuales hay que repartir únicamente una banca.
¿Cómo ha de asignarse esa banca?
Más allá de la definición que se brinde
del término \textsl{justicia} — siempre y cuando no haya preferencia arbitraria por ninguno de los partidos—, cualquiera de los candidatos
merecería la banca en igual proporción, y por ende cualquiera de las asignaciones posibles sería igual de justa o injusta.

Otro escenario, ubicado en las antípodas, sería el caso en el que la cantidad de bancas a repartir es
exactamente igual a la cantidad de votantes —podría ser el caso de un consejo barrial, por ejemplo—, en donde cada miembro tiene representatividad 
total.

Las formas y metodologías de encarar la resolución de esta problemática han ido variando en gran medida a lo largo de la 
historia. Algunas preguntas orientadoras a la hora de pensar distintos esquemas bajo los cuales intentar resolver el 
problema pueden ser:

\begin{itemize}[itemsep=0.6em]
    \item ¿Debe estar fijada de antemano la cantidad de bancas a asignarse?
    \item ¿Debe estar fijada de antemano la cantidad de individuos representados por cada miembro del parlamento?
    \item ¿Se debe intentar encontrar un sistema que optimice alguna métrica, como por ejemplo la cantidad de gente no representada?
\end{itemize}

Todas estas preguntas han llevado al diseño de diversos métodos, que posteriormente fueron tratados y analizados de una forma 
más general y sistemática por Balinski y Young en \cita{balinski1980theory}. Estos concluyeron, a través de los teoremas de Coherencia y de Quota,
que no existe ningún sistema de repartición de bancas que satisfaga todos los desiderata.

Inspirado por este resultado de imposibilidad, Geoffrey Grimmett plantea en \cita{Grimmett01042004} la idea 
de utilizar métodos aleatorizados para asignar bancas, en vistas de encontrar algún sistema que satisfaga
un conjunto de propiedades análogas al desiderata planteado por \cita{balinski1980theory} para métodos determinísticos.

\vspace{0.5em}

Con respecto al trabajo de esta tesis, haremos un recorrido por la historia de este problema, enunciando su definición
formal, los axiomas solicitados para considerar que un sistema es deseable, coherente o "justo", 
los desarrollos teóricos que se fueron produciendo alrededor de esto (principalmente por Balinski, Young,
Pukelsheim \cita{balinski1980theory, balinski1982fair, pukelsheim2017}). Daremos la definición de los llamados "\textsl{métodos de divisor}", junto 
con sus propiedades fundamentales, para luego enunciar el teorema de imposibilidad de Balinski-Young.
Esto nos servirá como motivación para definir la familia de métodos aleatorizados junto con sus respectivos axiomas deseables. 
Atravesaremos algunos ejemplos de métodos aleatorizados, analizaremos sus características y propiedades,
presentando el trabajo de \cita{correa2024monotonerandomizedapportionment}, y posteriormente trabajaremos alrededor
de algunas ideas que fueron surgiendo durante la ejecución de esta tesis, vinculadas a esta última familia de métodos.

Si los métodos aleatorizados son factibles de ser usados en escenarios electorales reales o no, es una 
inquietud que queda para el lector.

\subsection{Código de las pruebas computacionales}

Todo el código propio de las pruebas computacionales desarrolladas a lo largo de este trabajo está disponible en \href{https://github.com/lzanela/Tesis}{este repositorio de Github}.