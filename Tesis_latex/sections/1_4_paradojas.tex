\section{Paradojas y falencias de los métodos típicos}

El término "paradoja" para caracterizar aspectos de los métodos de \textit{apportionment} se utiliza no en un sentido lógico,
describiendo una proposición contradictoria en términos lógicos, sino en relación al "sentido común", describiendo
resultados que contradicen "lo que uno esperaría" de un sistema para distribuir bancas.

\subsection{Paradoja de Alabama}
Si bien el \textit{método de Hamilton} presenta una simpleza remarcable, esta simpleza lleva a resultados que pueden resultar insatisfactorios.
Luego del censo de 1880 en los Estados Unidos, el encargado de computar la asignación de bancas a cada estado realizó las cuentas para todos los 
congresos de entre 275 y 350 bancas, llegando a la conclusión de que si se repartían 299 bancas, Alabama recibiría 8, mientras que
repartiendo 300 bancas le corresponderían 7. Este escenario fue el primer ejemplo registrado en el cual se violó el axioma de 
\textit{House Monotonicity} \cite{caulfield2010apportioning}.

En general, la \textit{paradoja de Alabama} consiste en el no cumplimiento de la propiedad de \textit{House Monotonicity}.

A continuación, se muestra un ejemplo concreto de esta paradoja:

\begin{table}[h!]
    \centering
    \begin{tabular}{lcccccc}
    \hline
    \textit{Estado} & \textit{Población} &
    \multicolumn{2}{c}{\textit{Con 10 bancas}} &
    \multicolumn{2}{c}{\textit{Con 11 bancas}} \\
    \cline{3-6}
     & & \textit{Quota justa} & \textit{Bancas} &
         \textit{Quota justa} & \textit{Bancas} \\
    \hline
    A & 6 & 4.286 & 4 & 4.714 & 5 \\
    B & 6 & 4.286 & 4 & 4.714 & 5 \\
    C & 2 & 1.429 & 2 & 1.571 & 1 \\
    \hline
    \end{tabular}
    \caption{Distribución de bancas para diferentes tamaños de cámara extraído de \href{https://en.wikipedia.org/wiki/Apportionment_paradox#Alabama_paradox}{Wikipedia}.}
\end{table}
    

\vspace{0.25cm}

Un aspecto llamativo del \textit{método de Hamilton} es que no garantiza que si un partido recibe mayoría absoluta de los votos
entonces vaya a tener la mayoría absoluta de las bancas. Para ver esto, alcanza con tomar un escenario de 100 bancas a repartise entre 
3 partidos con el vector de votos $\vec{v} = (501,250,249)$. Las proporciones de votos obtenidas para este escenario resultan

$$\vec{q} = (0.501, 0.25, 0.249),$$

lo cual manifiesta la mayoría absoluta de votos para el partido $1$. 
Sin embargo, al utilizar el \textit{método de Hamilton} para repartir las bancas, en primer lugar se asignan:

\begin{itemize}[itemsep=0.6em]
    \item \( \lfloor 50.1 \rfloor = 50 \) bancas al partido 1,

    \item \( \lfloor 25 \rfloor = 25 \) bancas al partido 2,

    \item \( \lfloor 24.9 \rfloor = 24 \) bancas al partido 3.
\end{itemize}

La banca pendiente de ser repartida es asignada al partido 3, por tener un resto de $0.9$. Se observa entonces que el partido 1 no obtiene
la mayoría absoluta de las bancas a pesar de tener la mayoría absoluta de los votos.

%---------------------------------------------------------
\subsection{Paradoja de nuevos estados}
En la literatura hay diversos enunciados de esta paradoja, pero nos quedaremos con el de 
\cita{balinski1982fair}, que la relata de la siguiente forma:

Esta paradoja se produjo cuando, en 1907, Oklahoma se convirtió en un nuevo estado de los Estados Unidos.
Previamente, el congreso constaba de 386 bancas, asignadas a una población total de 74.562.608 habitantes, por lo que 
cada banca representaba aproximadamente 193.167 individuos. Como la población de Oklahoma era de alrededor de 1 millón
de habitantes, le corresponderían al menos 5 bancas. Por este motivo, se amplió la cantidad total de bancas a 391, de manera 
tal que Oklahoma obtuviese sus 5 bancas. 

Ante este escenario, lo esperable era que, al haber agregado las 5 bancas nuevas, el resto de los estados conserven las bancas que tenían.
No obstante, el método de Hamilton produjo la denominada \textit{paradoja de nuevos estados}: aplicándose con 391 bancas, 
le asignó 5 a Oklahoma, 4 a Maine y 37 a Nueva York; mientras que en el contexto anterior, sin la presencia de Oklahoma y con
un total de 386 bancas, le asignaba 38 bancas a Nueva York y 3 a Maine.
Esencialmente, al incorporarse Oklahoma agregando sus correspondientes bancas, Nueva York perdió una banca ante Maine. 

En resumen, la \textit{paradoja de nuevos estados} se produce cuando la incorporación de un estado nuevo, incorporando también
las respectivas bancas correspondientes, provoca alteraciones en la distribución de las bancas de los demás partidos.

%---------------------------------------------------------
\subsection{Paradoja de población}
En 1901, teniendo en cuenta la situación poblacional de Maine y Virginia, se calcularon las asignaciones
para estos estados: 9 bancas para Virginia y 4 para Maine; mientras que en 1900 las asignaciones eran de 10 bancas
para Virginia y 3 para Maine. Mirando un poco más en detalle la evolución de las poblaciones de estos estados en relación
a la evolución general del país, se observó que ambos estados habían aumentado sus poblaciones, y a pesar de que Virginia estaba 
creciendo más rápido que Maine, había perdido una banca frente a este estado.

Se analizaron las quotas, observándose que en 1900 eran de 9.599 para Virginia (redondeada hacia arriba por el método de Hamilton)
y de 3.595 para Maine (redondeada para abajo). La tasa de crecimiento de estos estados era de 1.07\% anual para Virginia y de 
0.67\% para Maine, contra una tasa general de 2.02\% del país entero. Con estos números, las quotas de estos estados en 1901 
habían decrecido a 9.509 para Virginia y 3.595\%, resultando en asignaciones de 9 y 4 bancas respectivamente \cita{balinski1982fair}.

El motivo por el cual se produjo esta paradoja está vinculado a que, a pesar de que el crecimiento de Virginia respecto a Maine
era mayor, como ambos estados tenían tasas de crecimiento que no llegaban a alcanzar la tasa de crecimiento del país, naturalmente
perderían bancas. Por ser Virginia un estado más grande, la diferencia entre su tasa de crecimiento y la tasa de crecimiento nacional
representaba, en términos absolutos, una cantidad más grande, que repercutió provocando que Virginia quede detrás de Maine en
términos de restos.

%-------------------------------------------------------------------
\subsection{Reflexiones acerca de los métodos típicos}
Recorriendo los ejemplos de métodos enunciados anteriormente vemos que la variedad de enfoques que se puede utilizar
para encarar la resolución de este problema es amplia, y no hay ninguna forma que resulte 
evidentemente más "justa" que las otras. Asimismo, la existencia de estas paradojas genera la sensación de que 
es necesario buscar métodos nuevos que puedan evitarlas sistemáticamente.
La pregunta es ¿cómo definir métodos nuevos?
Considerando las quotas, vemos que hay muchas formas según las cuales se las puede redondear 
para obtener una asignación de bancas: redondeando hacia arriba, hacia abajo, con redondeo tradicional, etc.
A su vez, no es trivial la forma en la que se definen las quotas. Si bien lo más intuitivo es 
dividir la cantidad de votos/habitantes por la \textit{Hare Quota}, esto no siempre redunda en
asignaciones satisfactorias. Estas dos observaciones nos alientan a considerar formas más generales
de definir métodos.
