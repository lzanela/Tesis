\section{Algunos métodos típicos}

Una cantidad que aparecerá recurrentemente en distintos métodos es la \textit{Hare Quota}, definida como

$$HQ(\vec{v}, H)=\frac{1}{H} \cdot \sum_{i=1}^{n}{v_i},$$

que podemos pensar como la cantidad promedio de votos representados por una banca, o el \textit{"precio justo"} (en
cantidad de votos) a pagar por una banca.

Usaremos el término \textit{quota} indistintamente para referirnos a la propiedad recién enunciada o al
cociente de la cantidad de votos de un partido por algún divisor $d$. En caso de no especificarse 
el divisor, se asume que es la \textit{Hare-Quota}.

A modo de raconto, vale la pena mencionar que en el siglo XVIII en Estados Unidos, en los orígenes de 
esta problemática, se pensaba en no dejar fija la cantidad de bancas a ocupar en el congreso, sino en fijar una
"razón de representación", es decir, buscar que haya "una banca cada $x$ personas". Naturalmente, surgen dos problemáticas
con este enfoque: por un lado, resulta ambigüo si se debe dividir el total de la población por $x$ para determinar el total
de bancas a repartir o si se debe dividir la población propia de cada estado por $x$ (definiendo también qué hacer con 
las partes fraccionarias resultantes). Estos dos posibles acercamientos producen, en general, resultados distintos.
Por otro lado, el hecho de no dejar fijo el \textit{house size} provoca que a medida que crece la población de un país con el
correr de los años, el house size pueda tender a infinito (obviamente, suponiendo que el divisor $x$ permanece constante a lo largo
del tiempo, y que los estados no varían sus legislaciones para distribuir las bancas). Para ejemplificar esto, basta apreciar el caso de los Estados Unidos,
en donde se estableció $30.000$ como divisor tras haber realizado su primer censo en 1790. La primera metodología utilizada
para asignar las bancas fue dividir la población de cada estado por este divisor, y redondear los valores obtenidos hacia abajo.
Por la cantidad de habitantes de ese entonces, el \textit{house size} resultante era de 112 miembros, como se puede apreciar
en la tabla 1.1.
Si se realizara el mismo procedimiento con las poblaciones actualizadas de los mismos 15 estados que existían en ese entonces
utilizando la información del último censo (2020) y el mismo divisor, el \textit{house size} resultante sería de 3.420, como 
se puede apreciar en la tabla 1.2.

\begin{table}[h!]
\centering
\begin{tabular}{lcc}
\hline
\textit{Estado} & \textit{Cociente (30.000)} & \textit{Asignación} \\
\hline
Connecticut      &  7.895  &  7  \\
Delaware         &  1.851  &  1  \\
Georgia          &  2.361  &  2  \\
Kentucky         &  2.290  &  2  \\
Maryland         &  9.284  &  9  \\
Massachusetts    & 15.844  & 15  \\
Nuevo Hampshire  &  4.727  &  4  \\
Nueva Jersey     &  5.986  &  5  \\
Nueva York       & 11.053  & 11  \\
Carolina del Norte & 11.784 & 11  \\
Pensilvania      & 14.429  & 14  \\
Rhode Island     &  2.282  &  2  \\
Carolina del Sur &  6.875  &  6  \\
Vermont          &  2.851  &  2  \\
Virginia         & 21.019  & 21  \\
\hline
\textbf{Total}   & \textbf{120.531} & \textbf{112} \\
\hline
\end{tabular}
\caption{Cociente y asignación de bancas para los 15 estados originarios de EEUU. Tabla extraída de \cita{balinski1982fair}}
\end{table}

\vspace{15em}

\begin{table}[h!]
\centering
\begin{tabular}{lrrr}
\hline
\textit{Estado} & \textit{Población (2020)} & \textit{Cociente (30.000)} & \textit{Asignación} \\
\hline
Connecticut            &  3\,605\,944 & 120.198 & 120 \\
Delaware               &    989\,948 &  32.998 &  32 \\
Georgia                & 10\,711\,908 & 357.063 & 357 \\
Kentucky               &  4\,505\,836 & 150.194 & 150 \\
Maryland               &  6\,185\,278 & 206.176 & 206 \\
Massachusetts          &  7\,029\,917 & 234.331 & 234 \\
Nuevo Hampshire        &  1\,377\,529 &  45.918 &  45 \\
Nueva Jersey           &  9\,288\,994 & 309.633 & 309 \\
Nueva York             & 20\,201\,249 & 673.375 & 673 \\
Carolina del Norte     & 10\,439\,388 & 347.980 & 347 \\
Pensilvania            & 13\,002\,700 & 433.423 & 433 \\
Rhode Island           &  1\,097\,379 &  36.579 &  36 \\
Carolina del Sur       &  5\,118\,425 & 170.615 & 170 \\
Vermont                &    643\,077 &  21.436 &  21 \\
Virginia               &  8\,631\,393 & 287.713 & 287 \\
\hline
\textbf{Total} & \textbf{102\,828\,965} & \textbf{3\,427.632} & \textbf{3\,420} \\
\hline
\end{tabular}
\caption{Actualización con datos del censo de 2020 para los 15 estados mencionados anteriormente. Datos extraídos de \href{https://www.census.gov/quickfacts/fact/table/CT/HSD410223}{Censo EE.UU.}
}
\end{table}
    

\newpage

%-------------------------------------------------------------------
\subsection{Método de Hamilton}
El primer sistema de distribución de bancas diseñado para el congreso estadounidense fue el \textit{método de Hamilton},
diseñado por Alexander Hamilton en 1792, también conocido como el \textit{método de restos mayores} (\textit{greatest remainders}).
Usualmente es considerado como el método más intuitivo. 
El método consiste en:

\begin{enumerate}[label=\textbf{Paso \arabic*.}, leftmargin=3.8em, itemsep=0.4em]
    \item \textbf{Cálculo de cuotas exactas:} 
    dividir los votos de cada partido por \( HQ(\vec{v}, H) \) y obtener \( q_i \coloneqq \frac{v_i}{HQ(\vec{v}, H)} = H\frac{v_i}{\sum_{i=1}^{n}{v_i}} \).

    \item \textbf{Asignación inicial:} 
    otorgar a cada partido \( i \) su \textit{lower quota}, \( \lfloor q_i \rfloor \).

    \item \textbf{Distribución de bancas restantes:} 
    ordenar los \textit{restos} \( p_i \coloneqq q_i - \lfloor q_i \rfloor \) de forma decreciente, 
    y asignar las bancas a los partidos con mayores restos.
\end{enumerate}

\vspace{0.25cm}

Un aspecto llamativo del \textit{método de Hamilton} es que no garantiza que si un partido recibe mayoría absoluta de los votos
entonces vaya a tener la mayoría absoluta de las bancas. Para ver esto, alcanza con tomar un escenario de 100 bancas a repartise entre 
3 partidos con el vector de votos $\vec{v} = (501,250,249)$. Las proporciones de votos obtenidas para este escenario resultan

$$\vec{q} = (0.501, 0.25, 0.249),$$

lo cual manifiesta la mayoría absoluta de votos para el partido $1$. 
Sin embargo, al utilizar el \textit{método de Hamilton} para repartir las bancas, en primer lugar se asignan:

\begin{itemize}[itemsep=0.6em]
    \item \( \lfloor 50.1 \rfloor = 50 \) bancas al partido 1,

    \item \( \lfloor 25 \rfloor = 25 \) bancas al partido 2,

    \item \( \lfloor 24.9 \rfloor = 24 \) bancas al partido 3.
\end{itemize}

La banca pendiente de ser repartida es asignada al partido 3, por tener un resto de $0.9$. Se observa entonces que el partido 1 no obtiene
la mayoría absoluta de las bancas a pesar de tener la mayoría absoluta de los votos.

- Se puede definir la Droop quota

%-------------------------------------------------------------------
\subsection{Método de Jefferson/D'Hont}

Existen diversas formas de definir el método D'Hont. Como es el sistema utilizado para la distribución
de bancas de la Honorable Cámara de Diputados de la Nación Argentina, se lo suele explicar en los
colegios secundarios a través del siguiente algoritmo:

\vspace{1em}

\begin{algorithm}[H]
\SetAlgoLined
 \textbf{Elaboración de tabla:} 
 Ordenar los partidos decrecientemente por cantidad de votos, poniendo como filas a los
 partidos\;
 \textbf{Cálculo de cocientes:} 
 En la primera columna numérica poner la cantidad de votos; en la segunda, la cantidad 
 de votos dividida por 2. Proceder de esta forma hasta llegar a la $H-$ésima columna, en la cual
 aparece, para cada partido, su cantidad de votos dividida por $H$\;
 \textbf{Asignación de bancas:} 
 Asignar las $H$ bancas a los partidos que posean los $H$ números más grandes en la tabla confeccionada\;
 {\Return La asignación resultante;}
 \caption{Método D'Hont}
\end{algorithm}

\begin{enumerate}[label=\textbf{Paso \arabic*.}, leftmargin=3.8em, itemsep=0.4em]
    \item \textbf{Elaboración de tabla:} 
    Ordenar los partidos decrecientemente por cantidad de votos, poniendo como filas a los
    partidos.

    \item \textbf{Cálculo de cocientes:} 
    En la primera columna numérica poner la cantidad de votos; en la segunda, la cantidad 
    de votos dividida por 2. Proceder de esta forma hasta llegar a la $H-$ésima columna, en la cual
    aparece, para cada partido, su cantidad de votos dividida por $H$.

    \item \textbf{Asignación de bancas:} 
    Asignar las $H$ bancas a los partidos que posean los $H$ números más grandes en la tabla confeccionada.
\end{enumerate}

Otra forma de definir este método parte del siguiente razonamiento:
si se quisiera asignar a cada partido $i$ el redondeo hacia abajo de la cantidad de bancas que le corresponden según la \textit{Hare-Quota}, 
$\lfloor \frac{v_i}{HQ(\vec{v}, H)} \rfloor$, esto tipicamente resultará en una asignación en la que quedan bancas libres.
Si se redondeara hacia arriba, la asignación excedería el total de bancas.

Luego, se podría buscar un divisor $d \leq HQ(\vec{v}, H)$ tal que redondeando hacia abajo estos cocientes, la suma total sea $H$:

\begin{equation}
    \sum_{i=1}^{n} \left\lfloor \frac{v_i}{d} \right\rfloor = H
    \label{eq:divisor_method}
\end{equation}

De esta forma, asignando a cada partido $i$ la parte entera de su cociente, $\lfloor \frac{v_i}{d} \rfloor$, se tiene una asignación válida.
De todos los divisores $d$ que sean solución a esta ecuación, usualmente se prefiere el más grande, a partir de pensar en este divisor
como el "precio por banca", medido en cantidad de votos, que paga cada partido \cita{balinski1976jefferson}, \cita{wilson_dhondt}.

La siguiente propiedad muestra que la asignación devuelta por el algoritmo coincide con la asignación producida a partir de la ecuación anterior:

\begin{proposition}
    Llamando $A(\vec{v}, H)$ a la asignación de bancas dada por el \textbf{algoritmo 1} y $A_d(\vec{v}, H)$ a la asignación dada por 
    $$A_d(\vec{v}, H) = (\left\lfloor \frac{v_1}{d} \right\rfloor, \dots, \left\lfloor \frac{v_n}{d} \right\rfloor),$$
    con $d$ solución de la ecuación~\eqref{eq:divisor_method}, se tiene que $A(\vec{v}, H) = A_d(\vec{v}, H)$.
    
    \begin{proof}
        Sin pérdida de generalidad, supongamos los partidos ordenados decrecientemente en cantidad de votos.
        Al seleccionar el número más grande de la tabla, $a_{11}$ (la cantidad de votos del partido 1, el partido con más votos), si se
        determina como precio por banca dicho valor, únicamente el partido 1 tiene la posibilidad de comprar una única banca.
        
        TO DO: TERMINAR DEMO
        
        Llamemos $a_k$ al número seleccionado en el paso $k \in \{1, \dots, H\}$ a partir de la tabla construida en el algoritmo, es decir,
        al $k$-ésimo valor más grande de la tabla. Digamos que es el valor ubicado en la posición $(ij)$ de la tabla: $a_k = \frac{v_i}{j}$.
        Esto implica que, si se utiliza $a_k$ como divisor, únicamente se llegan a asignar $k$ bancas:
        \begin{equation}
            \lfloor \frac{v_l}{a_k} \rfloor
                \begin{cases}
                    q_l, & \text{si } v_l \geq a_k, \\[6pt]
                    q_l - 1, & \text{si } v_l < a_k.
                \end{cases}
        \end{equation}
    \end{proof}
\end{proposition}

%-------------------------------------------------------------------
\subsection{Método de Webster/Saint-Laguë}
Para definir este método, notemos
\begin{equation}
    \llbracket x \rrbracket = 
        \begin{cases}
            \lfloor x \rfloor, & \text{si } x - \lfloor x \rfloor < \frac{1}{2}, \\[6pt]
            \lceil x \rceil, & \text{si } x - \lfloor x \rfloor > \frac{1}{2}, \\
            \{\lfloor x \rfloor, \lceil x \rceil\}, & \text{si } x - \lfloor x \rfloor = \frac{1}{2}
        \end{cases}
\end{equation} 
Este método surge de un razonamiento análogo al del método anterior: buscamos un divisor $d$ tal que al redondear según $[[\cdot]]$,
la asignación resultante sea válida, ie asigna $H$ bancas. Es decir, buscamos $d$ solución de 

\begin{equation}
    \sum_{i=1}^{n} \left\llbracket \frac{v_i}{d} \right\rrbracket = H
\end{equation}

Vale observar que, por la ambigüedad del caso $x - \lfloor x \rfloor = \frac{1}{2}$ en el redondeo, este método puede llegar a devolver
hasta $2^{\frac{s}{2}}$ asignaciones posibles para un $d$ fijo, con $s=|\{i \in [n]: \frac{v_i}{d} - \lfloor \frac{v_i}{d} \rfloor = \frac{1}{2}\}|$.

\subsection{Observaciones de los métodos típicos}

A continuación, veremos algunas situaciones atípicas que se han producido a lo largo de la historia ante la utilización
de estos métodos, motivando un análisis más profundo de las propiedades que cumplen, además de la definición de nuevos métodos.