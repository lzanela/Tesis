\section{Métodos de quota}


Citando las palabras de \cita{pukelsheim2017}, una manera de pensar a los métodos de quota es la siguiente:
\say{[...] los métodos de quota se pueden ver como procedimientos complementarios a los 
métodos de divisor. Los métodos de divisor fijan la regla de redondeo y ajustan el divisor. Los métodos
de quota fijan el divisor y ajustan la regla de redondeo.}

De esta forma, los métodos de quota se caracterizan por tener un divisor $Q > 0$ prefijado (típicamente la \textit{Hare-Quota}),
elegido por tener alguna característica persuasiva. Posteriormente, lo que se modifica es la regla de redondeo utilizada
para ajustar los cocientes resultantes al dividir los votos por $Q$.

Estos métodos constan de dos etapas:

\begin{itemize}
    \item En primer lugar se dividen las cantidades de votos de cada partido por el divisor prefijado $Q$, 
    y se asigna a cada partido la parte entera de su respectivo cociente, denominado \textit{quota inferior}: 
    $\lfloor \frac{v_i}{Q} \rfloor =\vcentcolon y_i$. De esta forma, se asignan $m \vcentcolon= y_+ = \sum_{i=1}^{n}{y_i}$ bancas.
    Cabe aclarar que el divisor $Q$ elegido debe ser tal que en esta primera etapa no se asignen más bancas de las que hay, $m \leq H$, 
    y que no falte asignar más de una banca por partido, $H - m \leq n$.
    
    \item Dado que en la primera etapa se asignaron $m$ bancas, resta asignar
    $H - m \in \{ 0, \dots , n \}$ bancas. Para esto, se establece un ordenamiento de los partidos de acuerdo a algún criterio que depende de
    los restos $\frac{v_i}{Q} - \lfloor \frac{v_i}{Q} \rfloor$, y se asignan las bancas restantes a los primeros $H-m$ partidos de este ordenamiento.
\end{itemize}

El método de quota más típico es el de Hamilton, enunciado en la sección 1.3, que utiliza como divisor $Q$ la \textit{Hare-Quota} $HQ(\vec{v}, H)$ 
definida anteriormente. Asigna las partes enteras de los cocientes resultantes, y luego simplemente ordena decrecientemente los restos 
$\frac{v_i}{Q} - \lfloor \frac{v_i}{Q} \rfloor$, asignando las $H-m$ bancas restantes a los $H-m$ partidos que tengan los restos más grandes.

Existe la posibilidad de que este método no entre en la segunda etapa de los métodos de quota. Para que ocurra esto, la primera etapa debe asignar
todas las bancas disponibles, lo cual sucede únicamente si los restos resultantes luego de la primera asignación son todos 0, ie 
$\lfloor \frac{v_i}{Q} \rfloor = \frac{v_i}{Q} \ \forall i \in [n]$. Esta última condición dice que las cantidades de votos de los partidos deben
ser todas múltiplos enteros de la \textit{Hare-Quota}, lo cual es sumamente improbable.

\subsection{Métodos shift-quota}

Los métodos shift-quota surgen a partir de pensar en que utilizando un divisor $d$ distinto de $HQ(\vec{v}, H)$ se puede llegar a 
un método de quota con mejores propiedades que el de Hamilton.
Se define la shift-quota con shift $s \in [-1; 1)$ como el divisor
\[
    d = Q(s) \vcentcolon= \frac{v_+}{H + s}.
\]

La utilidad de usar shifts $s \in [-1; 1)$ tiene que ver con buscar que, al utilizar la shift-quota y luego ajustar los residuos
con el criterio de "residuos más grandes", el método resultante sea válido. 

En efecto, la asignación de la primera etapa otorga 
$y_j = \lfloor \frac{v_j}{Q(s)} \rfloor = \lfloor \frac{v_j}{v_+} \cdot (H+s) \rfloor$ bancas al partido $j \in [n]$. De esta forma,

\begin{align*}
    y_+ &= \sum_{i=1}^{n}{\lfloor \frac{v_j}{v_+} \cdot (H+s) \rfloor} \\
        &\leq \sum_{i=1}^{n}{\frac{v_j}{v_+} \cdot (H+s)} \\
        &= H + s \\
        &< H + 1, \quad \text{pues s < 1},
\end{align*}

y por la integralidad de $y_+$ resulta $y_+ \leq H$.

Para la cota inferior, 
\begin{align*}
    y_+ &> \sum_{i=1}^{n}{\left[ \frac{v_j}{v_+} \cdot (H+s) - 1 \right]} \\
        &= (H + s) - n \\
        &\geq H - n - 1, \quad \text{pues s } \geq -1.
\end{align*}


\begin{proposition}
    (\textbf{Desigualdad max-min para métodos shift-quota})
    Consideremos un método shift-quota con ajuste por residuos más grandes y 
    shift $s \in [-1; 1)$ $A_s(\cdot, H)$. Entonces un vector de asignaciones
    $x \in \mathbb{N}^{n}_0(H) \in A_s(\vec{v}, H)$ si y solo si
    \[
        \max\limits_{j \in [n]} \left( \frac{v_j}{v_+} (H + s) - x_j \right) \leq \min\limits_{j \in [n]} \left( \frac{v_j}{v_+} (H + s) + 1 - x_j \right).
    \]
\end{proposition}

Esta desigualdad permite definir el intervalo de \textit{splits}

\[
    R(\vec{v}, x) \vcentcolon= \left[  \max\limits_{j \in [n]} \left( \frac{v_j}{v_+} (H + s) - x_j \right) \ ; \ \min\limits_{j \in [n]} \left( \frac{v_j}{v_+} (H + s) + 1 - x_j \right) \right].
\]

Este intervalo tiene los valores $r \in [0;1]$ que sirven como umbral, al considerar los restos,
para separar las partes fraccionarias de los partidos que reciben bancas extra
de los que no.


\begin{corollary}
    (\textbf{Métodos shift-quota y métodos de divisor estacionarios})
    Llamemos \textit{shQgrR}$_s$ al método shift-quota con shift $s \in [-1; 1)$
    y ajuste por residuos más grandes. Para todo $H \in \mathbb{N}$ y 
    $\vec{v} \in (0; \infty)^n$ existe $r^{*} \in [0;1]$ -- dependiente de 
    $s, H \text{ y } \vec{v}$ -- tal que \textit{shQgrR}$_s$ y el método de 
    divisor con regla de redondeo estacionario de parámetro $r^{*}$ tienen
    el mismo conjunto de asignaciones,
    \[
        shQgrR_s(\vec{v}, H) = DivSta_{r^{*}}(\vec{v}, H).
    \]  
\end{corollary}