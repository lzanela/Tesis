\section{Resultados principales del área}

En esta sección nos dedicaremos a analizar la utilidad y el grado de "justicia" con el que las familias
de métodos enunciadas anteriormente permiten resolver el problema de \textit{apportionment}. Esencialmente,
veremos que los métodos de divisor son los únicos métodos que no producen ninguna de las paradojas de Alabama, Población
y House Size mencionadas en la sección de ejemplos -- o, en otras palabras, son los únicos métodos
que verifican las propiedades de \textsl{monotonía de población}, \textsl{house monotonicity} y \textsl{coherencia} (que enunciaremos
en breve) --.
A su vez, el otro resultado importante que mencionaremos es que los métodos de divisor no satisfacen la propiedad 
\textit{quota}, concluyendo de esta forma que no pueden existir métodos verdaderamente satisfactorios, en el sentido
de que no produzcan paradojas y satisfagan la propiedad \textit{quota} simultáneamente.

Dado que los siguientes resultados fueron descubiertos y enunciados en relación a paradojas vinculadas a contextos de repartición
de bancas entre estados (y no entre partidos), los enunciaremos en esos mismos términos.

\vspace{1em}

\begin{proposition}
    (\textbf{Robustez de métodos de divisor ante \textsl{paradoja de población}})

    Un método de apportionment $A$ evita la \textsl{paradoja de población} si y solo si $A$ es un método de divisor.

    \begin{proof}
        Es fácil ver la vuelta: dado $A$ un método de divisor, supongamos que un estado $s_1$ aumenta su población \textsl{relativo} a otro estado 
        $s_2$ y el estado $s_2$ gana bancas. Entonces, la quota del estado $s_2$ debe haber pasado algún punto de salto, pero como 
        el aumento de $s_1$ es mayor \textsl{relativo} al de $s_2$, la quota de $s_1$ también debe haber avanzado, y por lo tanto no
        puede haber perdido bancas.

        La ida es más técnica y complicada y escapa a los fines de este trabajo, por lo que se puede revisar de \cita{balinski1982fair}.
    \end{proof}
\end{proposition}

\vspace{1em}

\begin{proposition}
    (\textbf{Robustez de métodos de divisor ante \textsl{paradoja de Alabama}})

    Todo método de divisor $A$ evita la \textsl{paradoja de Alabama}.

    \begin{proof}
        En los métodos de divisor, la cantidad de bancas obtenidas por un estado es equivalente a 
        la cantidad de puntos de salto que sobrepasó al dividir su cantidad de votos por cierto divisor $d$.
        Si la cantidad total de bancas a repartir aumenta, el divisor $d$ debe decrecer, y por lo tanto las quotas
        $\frac{v_i}{d}$ aumentan. Esto implica que ningún estado puede perder bancas.
    \end{proof}
\end{proposition}

\vspace{1em}

\begin{proposition}
    (\textbf{Falencia de métodos de divisor ante propiedad \textit{quota}})
    No existe método de divisor $A$ que verifique la propiedad \textit{quota} para todo problema.
\end{proposition}

\vspace{1em}

Introducimos la siguiente notación, que facilitará la escritura de las siguientes propiedades:
dado un vector $x \in \mathbb{R}^n$ y un subconjunto de índices $I \subset \{1, \dots , n\}$, notamos

\[
    x_I \coloneq \sum_{i \in I}{x_i}.
\]

Esto nos permitirá formalizar un concepto planteado por \cita{balinski1982fair}, con el cual caracterizan la relación
que ha de existir entre las partes y el todo de una solución: 
\say{
    Un principio inherente a cualquier división justa es que toda parte de una división justa debe ser justa. 
    A la inversa, la solución completa debe ser reconstruible a partir de concatenar soluciones de problemas parciales.
    El todo y sus partes deben encajar juntos de forma coherente.
}
Introducimos entonces la definición de \textsl{coherencia}.

\begin{definition}
(\textbf{Coherencia})
Un método de \textit{apportionment} $A$ verifica la propiedad de \textit{\textbf{Coherencia}} si para todo $n \in \mathbb{N}$ y 
para todo $\vec{v} \in \mathbb{R}_{\geq 0}^{n}$, todo vector de asignaciones $x \in A(\vec{v}, x_+)$ satisface, para cualquier subconjunto
de partidos $I \subset \{ 1, \dots , n\}$, las siguientes propiedades:

\begin{enumerate}
    \item (\textit{Coherencia de subproblemas}) 
    $$(x_i)_{i \in I} \in A((\vec{v}_i)_{i \in I}, x_I)$$

    \item (\textit{Concatenación de soluciones parciales})
    $$\text{para todo }(y_i)_{i \in I} \in A((\vec{v}_i)_{i \in I}, x_I) \text{ y } 
    \text{para todo }(z_i)_{i \in I^{c}} \in A((\vec{v}_i)_{i \in I^{c}}, x_{I^{c}})$$ 
    se tiene que $((y_i)_{i \in I}, (z_i)_{i \in I^{c}}) \in A(\vec{v}, x_+)$.
\end{enumerate}

Esencialmente, la propiedad de \textsl{coherencia} establece que, dada una asignación $a$ de bancas a partir de un método $A$, 
si se utiliza el método sobre un subconjunto $S$ de partidos con un tamaño de cámara de $\sum_{i \in S}{a_i}$, 
entonces la asignación correspondiente $S$ debe permanecer igual que antes.

\end{definition}

\begin{theorem}
    (\textbf{Teorema de coherencia})

    Un método satisface la propiedad de \textsl{coherencia} si y solo si es un método de divisor.

    \begin{proof}
        La prueba de la ida se realiza analizando una clase de métodos más amplia, los métodos rank-index, para
        luego construir una secuencia de saltos que permite concluir que $A$ es un método de divisor.
        
        Debido a su tecnicidad, no realizaremos esta demostración en el presente trabajo.
        %Debido a su tecnicidad,
        %esta demostración fue realizada originalmente por \cita{balinski1980theory}. 
        %Emulamos una demostración más novedosa, de \cita{PALOMARES201611} en el apéndice A.

        %Para la vuelta, sea $A$ un método de divisor. Para verificar su \textit{coherencia}, basta ver que:
        %
        %si existe divisor $D > 0$ para el cual $x \in A(\vec{v}, x_+)$, entonces, dado $I \subset \{ 1, \dots , n\}$, 
        %dicho divisor también sirve para $(x_i)_{i \in I} \in A((\vec{v}_i)_{i \in I}, x_I)$, por lo que la 
        %\textit{coherencia de subproblemas} se verifica de forma inmediata.
        %
        %Por otro lado, si se tiene $(y_i)_{i \in I} \in A((\vec{v}_i)_{i \in I}, x_I) \text{ y } 
        %(z_i)_{i \in I^{c}} \in A((\vec{v}_i)_{i \in I^{c}}, x_{I^{c}})$, es por

    \end{proof}
\end{theorem}

\begin{corollary}
    (\textbf{Robustez de métodos de divisor ante \textsl{paradoja de nuevos estados}})
    Todo método de divisor $A$ evita la \textsl{paradoja de nuevos estados}.

    \begin{proof}
        Para ver esto, pensemos en la \textsl{paradoja de nuevos estados} al revés de como fue planteada en la sección de paradojas: 
        en lugar de considerar que se agrega un estado nuevo junto con la cantidad correspondiente de bancas según redondeando su quota justa (osea,
        la quota obtenida al dividir la población del nuevo estado por la Hare-Quota), consideremos un estado $s$ que desaparece, llevándose consigo
        la cantidad $a_s$ de bancas que tenía. Justamente, como sus bancas desaparecen, el divisor $D$ que estaba siendo utilizado antes de la desaparición del estado $s$ sigue 
        sirviendo para los estados restantes, y al considerar $\llbracket \frac{v_i}{D} \rrbracket$ para $i \in [n] \setminus \{s\}$, se tiene que la asignación
        que se tenía antes para los estados distintos de $s$ sigue sirviendo. Dado que las asignaciones permanecen constantes ante la aparición/desaparición
        de un estado con sus respectivas bancas, no se produce la \textsl{paradoja de nuevos estados}.
    \end{proof}
\end{corollary}