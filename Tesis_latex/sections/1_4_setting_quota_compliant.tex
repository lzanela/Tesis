\section{Métodos quota-compliant aleatorizados}

Sean $n \in \mathbb{N}$ con $[n] = \{ 1, \dots, n\}$ un conjunto de partidos, $H \in \mathbb{N}$ cierta cantidad de bancas a repartir.
\vspace{0.25cm}

\textbullet \ $ \vec{v} = (v_1, \dots, v_n) \in \mathbb{N}^n$ vector de votos recibidos por cada partido,

\textbullet \ $\vec{q} = (q_1, \dots, q_n)$ vector de proporciones, con $q_i = H\frac{v_i}{\sum_{i=1}^{n}{v_i}}$ y $\sum_{i=1}^{n}{q_i} = H$

\textbullet \ $\vec{p} = (p_1, \dots, p_n)$ vector de restos, con $p_i = q_i - \lfloor q_i \rfloor$ y $ \sum_{i=1}^{n}{p_i} = K$ la cantidad de bancas pendientes de ser repartidas. Cabe aclarar que $ 0 \leq p_i < 1 \ \forall \ 1 \leq i \leq n$

\textbullet \ Para un conjunto $S$ y $k \in \mathbb{N}$, definimos $ \binom{S}{k} := \{ T \subseteq S \ / \ |T| = k \}$


\begin{definition} Un método de \textit{apportionment a} es una función que toma un vector de votos $\vec{v}$, una cantidad de bancas $H$, y retorna un vector aleatorio $a(\vec{v}, H) \in \mathbb{N}^n$ que suma H, e indica la cantidad de bancas asignadas a cada partido.

Similarmente, un método de \textit{redondeo} $r$ es una función que toma un vector de residuos $\vec{p} \in [0,1)^{n}$ que suma $k$, y le asigna un conjunto aleatorio $r(\vec{p}) \in \binom{[n]}{k}$, es decir, el conjunto de los $k$ partidos seleccionados. Además, la probabilidad de que el partido $i$ esté en el conjunto seleccionado $r(\vec{p})$ debe ser igual a $p_i \ \forall \ i \in [n]$. 
\end{definition}

\vspace{0.25cm}

\noindent Buscamos $X_i \in \{ 0,1 \} \ / \ \sum_{i=1}^{n}{X_i} = K \ \wedge \ \mathbb{E}[X_i] = q_i$. La distribución del vector $X = (X_1, \dots, X_n$ estará dada por algún algoritmo aleatorizado, es decir, $a(\vec{v}, H) = (X_1, \dots, X_n)$ para algún método $a$.