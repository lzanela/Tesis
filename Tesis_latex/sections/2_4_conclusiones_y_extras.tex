\section{Conclusiones y algunos resultados extra}

Como ya hemos mencionado al comienzo de la sección anterior, la conjunción del \textit{Teorema de Coherencia} y 
la propiedad de que los métodos de divisor no satisfacen \textit{quota} funcionan, de alguna forma, como un resultado de 
imposibilidad: no existe método determinístico de \textit{apportionment} que sea coherente (en particular, que verifique 
las propiedades implicadas por coherencia) y verifique la propiedad \textit{quota} simultáneamente.

El término "determinístico" en la frase anterior no es en vano: esta suerte de "Teorema de imposibilidad" no
extraer conclusiones sobre métodos aleatorizados. Si bien las propiedades deseables que enunciamos en la subsección 
\ref{subsec:props_deseables} no hacen referencia a métodos que contemplen algún factor de aleatoriedad, veremos 
cómo definir formalmente esta familia de métodos, junto con propiedades análogas, para poder realizar un análisis
sistemático de esta familia de métodos.
No obstante, antes de analizar este universo de métodos, revisamos algunos conceptos y resultados extra sobre 
los métodos determinísticos.

Uno de los conceptos interesantes que introduce \cita{pukkelsheim2017} es el de \textit{mayoración},
que permite entender la relación entre distintos métodos de \textit{apportionment} a la hora de 
favorecer o perjudicar a los partidos más grandes. Se basa, esencialmente, en establecer una relación
de orden entre métodos, definiendo que un método $A$ mayora ($\succeq$) a otro método $B$ si, al 
considerar a los partidos ordenados decrecientemente por proporción de votos ($w_1 \geq w_2 \geq \dots \geq w_n$), 
el método $A$ asigna un total de bancas mayor que $B$ a cualquier subconjunto $I$ de $k$ partidos más grandes
($I = \{ 1, \dots, k\} \ \wedge x \in A(\vec{w}, H), y \in B(\vec{w}, H) \ \implies x_1 + \dots + x_k \geq y_1 + \dots + y_k$).

Otra cuestión interesante que analiza \cita{pukkelsheim2017} es cómo extender de forma coherente el método de apportionment natural para
2 partidos: utilizar redondeo estándar $\llbracket \cdot \rrbracket$ (recordamos que, en caso de que la parte fraccionaria del número a redondear fuese igual a $\frac{1}{2}$, 
el redondeo estándar retornaba las dos opciones posibles:
redondeo hacia arriba o hacia abajo). Es decir, si se tienen dos partidos con cantidades de votos dadas por $(v_1, v_2) \in \mathbb{R}_{> 0}$ y 
$H$ bancas a repartir, asignarles $(x_1, x_2)$ bancas respectivamente, con 

\[
    x_1 \in \left\llbracket \frac{v_1}{v_1 + v_2} H \right\rrbracket \quad \wedge \quad x_2 \in \left\llbracket \frac{v_2}{v_1 + v_2} H \right\rrbracket
\]

El análisis de esta situación llega a la conclusión de que el único método que extiende esta asignación de 2 partidos en forma coherente es el método
de divisor con redondeo estándar: el de Webster/Saint-Laguë.