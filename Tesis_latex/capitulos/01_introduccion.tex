\chapter{Introducción}

Según \cita{knuth1984texbook}, LaTeX es un sistema tipográfico poderoso.

En física, la teoría de la relatividad fue presentada por \cita{einstein1905}.

En física, la teoría de la relatividad fue presentada por \cita{einstein1905}.

Another category of methods may appeal to the gambling man. As one example, construct a roulette wheel divided into fifty slots, one for
each state, the size of each slot being exactly proportional to the population of the state. Spin the wheel and drop a small ball onto it: the state at which it
comes to rest4 'wins5 5 and is awarded one seat. Do this 435 times consecutively and the house is apportioned. The method is perfectly unbiased:
every state is treated fairly; none can complain that the method discriminates against it.
An alternative and perhaps preferable scheme is to compare the
quotas of the states and first give to each the whole number in its quota;
then use a roulette wheel, with each slot proportional to a state's remainder,
to distribute the seats left over.
Which of all possible methods is fairest! To answer this question,
standards or principles must be formulated by which methods can be
evaluated. History itself points to such principles. What are the methods
that avoid the population paradox, the Alabama paradox, the new states
paradox? What are the methods that always stay within the quota? What are
the methods that systematically favor neither the large states at the expense
of the small nor the small at the expense of the large? Or, for proportional
representation systems, which methods discourage the splintering of large
parties into smaller ones?
Each of these questions captures a principle of apportionment. They
have repeatedly been used in history to judge the merits of competing
proposals. In a word, they are the guides to those methods that feel right,
that are fair, that are indeed proportional. \cita{balinski1982fair} discusses various methods of apportionment in detail.
