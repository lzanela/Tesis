\section{Ejemplos de cosas}
\begin{itemize}
    \item Ejemplo de population paradox con Hamilton: \cita{balinski1982fair} pág. 69, tabla 8.1
    \item Desarrollo de mayoración: \cita{pukelsheim2017} cap. 8
\end{itemize}

\section{Temas a charlar en la reu}
\begin{itemize}
    \item Definir nombres de las secciones para las pruebas que fui haciendo. 
    orden lexicográfico? Función objetivo lineal con pesos decrecientes por orden lexicográfico sobre subconjuntos de cardinal k  
    \item Nombre de la primera sección: Introducción al problema de \textit{apportionment}
    Se introduce también el trabajo de la tesis, así que capaz debería cambiar el título.
\end{itemize}

\section{Títulos posibles}
\begin{itemize}
    \item "Una aproximación al problema de apportionment: métodos aleatorizados"
    \item "Una aproximación al problema de apportionment: métodos aleatorizados para distribución de bancas del congreso"
    \item "Una aproximación al problema de apportionment: análisis y propuestas de sistemas de distribución 
    de bancas aleatorizados"
    \item "Una aproximación al problema de apportionment a través del análisis y propuestas de sistemas de distribución 
    de bancas aleatorizados"
    \item "Una aproximación al problema de apportionment: distribuyendo las bancas del congreso aleatoriamente"
    \item "Métodos aleatorizados de asignación de bancas y optimización lineal junto a algunos resultados de imposibilidad"
    \item "Asignación aleatorizada de bancas y algunos resultados de imposibilidad"
\end{itemize}


\section{Cosas a corregir/agregar}
\begin{itemize}
    \item Ser más concluyente con el teorema de imposibilidad de Balinski y Young
    \item Demostración por inducción de que Brewer es ex-ante
    \item Demostración de que los métodos de divisor satisfacen anonimidad, balance, concordancia, decencia y exactitud.
    \item Agregar definición del intervalo D de divisores posibles luego de la desigualdad max-min.
    \item Relación con el problema del stopping game. Ver Balinski y Young.
    \item De todos los métodos de divisor, el de Webster es el que tiene mayor probabilidad de satisfacer quota.
    \item Buscar otra caracterización del método Saint-Laguë y demostrar equivalencia con el que ya está enunciado.
    \item D'Hondt favorece mayorías
    \item Métodos determinísticos optimales, distintas métricas de optimalidad y equivalencias con métodos existentes
    \item ¿Por qué selection monotonicity no se plantea en términos de los votos en lugar de los restos?
\end{itemize}

\section{Cosas sacadas (se pueden volver a agregar)}
    