\section{Ejemplos de cosas}
\begin{itemize}
    \item Ejemplo de population paradox con Hamilton: \cita{balinski1982fair} pág. 69, tabla 8.1
    \item Desarrollo de mayoración: \cita{pukkelsheim2017} cap. 8
\end{itemize}


\section{Títulos posibles}
\begin{itemize}
    \item "Una aproximación al problema de apportionment: métodos aleatorizados"
    \item \item "Una aproximación al problema de apportionment: métodos aleatorizados para distribución de bancas del congreso"
    \item "Una aproximación al problema de apportionment: análisis y propuestas de sistemas de distribución 
    de bancas aleatorizados"
    \item "Una aproximación al problema de apportionment a través del análisis y propuestas de sistemas de distribución 
    de bancas aleatorizados"
    \item "Una aproximación al problema de apportionment: distribuyendo las bancas del congreso aleatoriamente"
\end{itemize}


\section{Cosas a corregir/agregar}
\begin{itemize}
    \item Terminar gráfico de elecciones en Apportia
    \item $D-$quota para referirse a $\frac{v_i}{D}$
    \item D'Hont favorece mayorías
    \item Métodos determinísticos optimales, distintas métricas de optimalidad y equivalencias con métodos existentes
    \item ¿Por qué selection monotonicity no se plantea en términos de los votos en lugar de los restos?
    \item 
\end{itemize}